

\chapter{Atividades de Ensino e de Orientações}\label{cap:ensino}

No capítulo anterior, Seção \ref{sec:profissionalAntes}, apresentei de forma resumida as disciplinas ministradas nos Centros Universitários Senac e FEI. Nestas instituições ministrei muitas turmas, principalmente nas disciplinas de Programação e de Engenharia de Software (ES). Como não possuo os comprovantes das turmas por semestre nesse período, não apresentarei essa importante fase de minha vida acadêmica. Assim, neste capítulo irei detalhar as atividades de ensino realizadas apenas na UFABC. Além disso, irei apresentar as orientações em diferentes cursos e modalidades que realizei até o momento.

\noindent
\textbf{Notações:} Os períodos letivos na UFABC estão organizados em três quadrimestres anuais, por exemplo 2008.1, 2008.2 e 2008.3. Cada quadrimestre possui 12 semanas. A quantidade $x$ de turmas que ministrei de uma mesma disciplina em um quadrimestre será denotada por $x$T. Por exemplo, 2009.1.2T denota que ministrei duas turmas de uma mesma disciplina no primeiro quadrimestre de 2009.

\section{Disciplinas de Graduação}

A seguir são descritas as disciplinas ministradas no curso Bacharelado em Ciência e Tecnologia (BC\&T) na UFABC:
\begin{description}
    \item [Linguagens de Programação:] 2008.1.1T;
    \item [Geometria Analítica:] 2010.3.3T;
    \item [Bases Computacionais da Ciência:] 2009.1.2T, 2012.2.2T, 2013.2.2T, 2020.2.1T, 2021.3.1T, 
    \item [Processamento da Informação:] 2011.1.3T, 2012.1.5T, 2014.3.2T, 2015.1.2T, 2015.3.2T, 2016.1.1T, 2016.1.2T-EaD, 2016.2.2T-EaD, 2016.3.1T-EaD, 2017.1.2T-EaD, 2017.2.2T-EaD, 2018.1.2T-EaD, 2018.2.1T-EaD, 2018.3.2T-EaD, 2019.1.3T-EaD, 2020.1.4T-EaD, 2021.2.3T, 2022.2.4T, 
\end{description}

A seguir são descritas as disciplinas ministradas no curso de  Bacharelado em Ciência da Computação (BCC) na UFABC:
\begin{description}
    \item [Engenharia de Software:] 2009.1.1T, 2009.3.1T, 2010.1.1T, 2013.1.3T, 2014.1.1T, 2015.1.1T, 2016.1.1T, 2017.1.1T, 2019.1.1T, 
    \item [Sistemas Multimídia:] 2009.3.1T, 
    \item [Sistemas de Informação:] 2008.3.2T, 2014.1.1T
    \item [Processamento Digital de Imagens:] 2011.2.1T, 2016.3.1T, 2019.3.1T, 2020.1.1T, 2021.1.1T, 2021.2.1T, 2022.1.1T,  
    \item [Programação Orientada a Objetos:] 2008.2.1T, 2011.3.2T, 2012.3.2T, 2015.3.2T, 
    \item [Processamento de Imagens Utilizando GPU:] 2016.3.1T;
    \item [Programação Estruturada:] 2022.3.5T;
\end{description}

\section{Disciplinas de Pós-Graduação Stricto Sensu}

A seguir são descritas as disciplinas ministradas no curso de Pós-Graduação em Ciência da Computação na UFABC:

\begin{description}
    \item [Visão Computacional e Processamento de Imagens:] 2013.3.1T, 2018.3.1T, 2019.3.1T, 2021.1.1T, 2022.1.1T, 
\end{description}

A seguir são descritas as disciplinas ministradas no curso de Pós-Graduação em Engenharia da Informação na UFABC:
\begin{description}
    \item[Introdução à Engenharia da Informação:] 2009.1.1T;
    \item[Processamento e Visualização de Imagens:] 2009.2.1T;
    \item[Metodologias para Modelagem de Sistemas:] 2010.1.1T;
\end{description}

\section{Disciplinas de Pós-Graduação Lato Sensu}

Nesta seção e descrita a disciplina ministrada no curso de pós-graduação Lato Sensu em Tecnologias e Sistemas de Informação da UFABC em parceria com a Universidade Aberta do Brasil no modelo de Ensino a Distância (EAD):

\begin{description}
    \item [Sistemas Corporativos e Informação:] 2013.2.4T.
\end{description}

\section{Programa de Ensino e Aprendizagem Tutorial da UFABC}

O \href{https://prograd.ufabc.edu.br/peat}{Programa de Ensino e Aprendizagem Tutorial} (PEAT) foi iniciado na UFABC em sua criação, em 2006, e tem como objetivo auxiliar os ingressantes na vida acadêmica. Essa iniciativa é importante, principalmente pelo projeto pedagógico inovador da UFABC, onde o estudante ingressa em bacharelados interdisciplinas e depois escolhe cursos específicos, como ciência da computação, conforme desempenho e/ou motivações em disciplinas/áreas específicas. Se o estudante não planejar bem as escolhas das disciplinas, poderá gastar mais tempo para integralizar o curso específico pretendido. Quando eu orientei os estudantes ingressantes em 2008 e 2009, fiz reuniões periódicas para explicar como funciona a vida acadêmica na UFABC e tirar dúvidas individuais sobre disciplinas, grades curriculares, matrícula, desempenho acadêmico, etc. Atualmente o PEAT está sendo reformulado para melhor atender os ingressantes.


\section{Orientações e supervisões em andamento}

Nesta seção apresento as três orientações em andamento no Mestrado, além de uma orientação no TCC da graduação, ambas em Ciência da Computação na UFABC.

\subsection{Dissertação de Mestrado}

\input{texLattes/OrientacoesAndamentoMestrado.tex}

\subsection{Trabalho de conclusão de curso de graduação}

\input{texLattes/OrientacoesAndamentoGraduacao.tex}

\section{Orientações e supervisões concluídas}

\subsection{Dissertação de mestrado}

Nesta seção apresento as cinco orientações de mestrado concluídas na UFABC.

\input{texLattes/OrientacoesMestrado.tex}

\subsection{Monografia de conclusão de curso de aperfeiçoamento/especialização}

Na UFABC atuei como coordenador de tutores e vice-coordenador, além de co-orientar um TCC. No Centro Universitário Senac orientei seis TCCs em especializações entre 2004 e 2005, descritos a seguir:

\input{texLattes/OrientacoesOutrasAper.tex}

\subsection{Trabalho de conclusão de curso de graduação}

Atuei como orientador de TCC no Centro Universitário Senac (dois no total) e também na UFABC nos cursos de Bacharelado em Ciência da Computação (quatro no total) e Engenharia da Informação (apenas um TCC), descritos a seguir:

\input{texLattes/OrientacoesOutrasTrab.tex}

\subsection{Iniciação científica}

Atuei como orientador de estudantes de Iniciação Científica (IC) no Centro Universitário Senac (três no total) e também na UFABC (23 no total) nos cursos de Bacharelado em Ciência e Tecnologia (BC\&T) e de Bacharelado em Ciência da Computação (BCC). Como os comprovantes fornecidos pelas IES, as vezes, não especificam as fontes de fomentos às bolsas, omiti essa informação em alguns casos. Na UFABC, os comprovantes também não especificaram de qual curso é o aluno, desta forma deixei todos com BC\&T, curso ingressante de todos os estudantes que orientei. Além disso, na UFABC existem programas de fomentos, em geral, em parceria com o CNPq: Pesquisando Desde o Primeiro Dia (PDPD), para incentivar os ingressantes na pesquisa; Jovens Talentos para a Ciência (JTC); Programa Institucional de Bolsas de Iniciação Científica (PIBIC); PIBIC-AF (Ações Afirmativas), dentre outros. Todas essas orientações estão descritas a seguir:

\input{texLattes/OrientacoesOutrasInic.tex}

\subsection{Orientações de outra natureza}

Em 2009 fui procurado pelo doutorando Luciano R. Neves, especialista em otorrinolaringologia, com interesse em construir um Sistema Biomecânicos para Diagnóstico Auxiliado por Computador \cite{2009:Cuzziol.Marques.ea,2010:Neves.Marques.ea}. Esse especialista custeou informalmente duas bolsas de IC durante um ano, descritas a seguir:

\input{texLattes/OrientacoesOutrasOrie.tex}

\section{Considerações Finais}

Percebi o interesse no ensino ao ajudar estudantes do ensino fundamental em aulas particulares de matemática, no final da década de 80, no início da minha graduação na UFES. Continuei esse interesse ao ministrar aulas de noções de computação para estudantes e funcionários da USP, durante o meu mestrado. Porém, o grande aprendizado como professor foi ao ministrar muitas turmas de Programação e de Engenharia de Software (ES) nos Centros Universitário Senac e FEI iniciado no final do meu doutorado. Por exemplo, lembro que era comum eu ministrar cinco turmas de ES por semestre e isso ocorreu durante seis anos, até 2007. Por questões pessoais de ter que morar em São Paulo, após concluir o doutorado continuei nessas excelentes IES particulares por mais cinco anos até surgir a oportunidade de prestar o concurso público na UFABC (esses relatos foram introduzidos no capítulo anterior). 

A partir de janeiro de 2008, já como professor da UFABC, o meu tempo dedicado ao ensino e à pesquisa ficou mais balanceado, algo muito desejado por mim. As Atividades de Ensino realizadas por mim na UFABC foram detalhadas neste Capítulo \ref{cap:ensino}. Porém, vale destacar que desde 2008 procurei realizar atividades envolvendo Ensino, Pesquisa (detalhados no Capítulo \ref{cap:pesquisa}), Extensão (detalhados no Capítulo \ref{cap:extensao}) e Administração (detalhados no Capítulo \ref{cap:admin}). Por exemplo, a Especialização em Tecnologias e Sistemas de Informação (TSI), foi a primeira especialização oferecida pela UFABC em 2010, em parceria com o programa Universidade Aberta do Brasil, vinculada ao MEC (UAB/MEC), onde ministrei uma disciplina e fui coordenador dos tutores nas edições de 2010, 2012 e 2014. Em 2017 foram ofertada turmas somente com recursos da UFABC e não tivemos tutores. A partir desta edição de 2017 estou como vice-coordenador. Todas essas edições tivemos muitos candidatos inscritos e para facilitar o processo de avaliação foi necessário desenvolver um software, chamado MCTest, \textit{Multiple-Choice Test}, para correção automática das provas utilizando Processamento Digital de Imagens. Além do TSI, a Escola Preparatória da UFABC, um pré-vestibular gratuito, utiliza esse serviço em seu processo seletivo. 

O MCTest foi codificado integralmente por mim, é um software de código aberto e gratuito no \href{https://github.com/fzampirolli/mctest}{GitHub} e está em sua quinta versão, agora como um serviço Web. Na UFABC o MCTest foi implantado em \url{http://mctest.ufabc.edu.br} para professores. Recentemente, o serviço é utilizando principalmente em disciplinas envolvendo programação, com integração do \textit{plugin} VPL (\textit{\href{https://vpl.dis.ulpgc.es}{Virtual Programming lab for Moodle}}). Com isso, o MCTest permite criar questões parametrizadas de exercícios de programação, com correção automática no Moodle. 

Motivado para criar novas funcionalidade no MCTest, consequentemente para novas contribuições ao Estado da Arte, consegui um projeto de auxílio à pesquisa na FAPESP, ministrei e coordenei por vários anos as disciplinas BC\&T: Bases Computacionais da Ciência; Processamento da Informação; e Programação Estruturada. Nessas coordenações tentei unificar conteúdos e principalmente avaliações nas dezenas de turmas ofertadas  para 2.000 estudantes em média por ano em cada disciplina.
\
O desenvolvimento do MCTest gerou várias publicações, com dezenas de colaboradores e irei detalhar mais essa pesquisa nos próximos capítulos.
