
\chapter{Atividades de Pesquisa}\label{cap:pesquisa}

Este capítulo apresenta as atividades de pesquisa realizadas principalmente após o meu ingresso como professor na UFABC, em 2008. As minhas pesquisas realizadas durante a minha formação acadêmica e antes de ingressar na UFABC foram introduzidas no Capítulo \ref{cap:introducao}. Apresento também comitês e revisor de eventos, pareceres ad hoc, palestras, participações em eventos e bancas e finalmente as publicações científicas e prêmios.

\section{Projetos de Pesquisa}\label{sec:projetos}

Nesta seção apresento um resumo dos projetos de pesquisa em que participei, dos colaboradores, financiadores, etc. As publicações resultantes desses projetos estão descritas no final deste capítulo.

\subsection{Coordenação de Projetos de Pesquisa}

\subsubsection{Modelagem de objetos usando morfologia matemática e grafos de vizinhança}

\noindent\textbf{Resumo:} A modelagem de objetos usando morfologia matemática e grafos de vizinhança, além da validação de novos conceitos e testes de eficiência, serão trabalhos neste projeto de pesquisa. Propomos dar continuidade aos estudos realizados de operadores morfológicos usando grafos de vizinhança e aplicar em problemas de processamento de imagens, destacando a segmentação de células em biologia.

\begin{description}
    \item[Vigência:] Mai/2010 - Abr/2012;
    \item[Auxílio:] Financiamento FAPESP, chamada Auxílio à Pesquisa;
    \item[Processo:] \href{https://bv.fapesp.br/pt/auxilios/28430/modelagem-de-objetos-usando-morfologia-matematica-e-grafos-de-vizinhanca/}{2009/14430-1};
    \item[Coordenador:] Prof. Francisco de Assis Zampirolli.
    %\item[Membros da Equipe::] 
\end{description}

O Auxílio financeiro disponibilizado pela FAPESP ajudou em todas as publicações realizadas durante a vigência do projeto. Além disso, foi possível comprar equipamentos para a UFABC, como um servidor onde hospedo o portal \url{http://vision.ufabc.edu.br}. 

\subsubsection{Um sistema universal para geração e correção automática de questões parametrizadas}

\noindent\textbf{Resumo:} Apresentamos uma solução inovadora para simplificar a geração e correção de questões parametrizadas para serem aplicadas em exames em papel, podendo ser reutilizáveis em várias instituições de ensino. Foi desenvolvido inicialmente um protótipo \textit{online} para validar conceitos, resultando inicialmente em várias propostas de contribuições científicas na área de educação, sistemas de informação, visão computacional e computação de alto desempenho. Além disso, este protótipo está em operação e auxilia professores de diversas instituições na geração e correção de exames. Esse projeto de pesquisa visa fomentar a divulgação dessas contribuições científicas e fazer melhorias na usabilidade das interfaces gráficas existentes.

\begin{description}
    \item[Vigência:] Set/2019 - Ago/2021;
    \item[Auxílio:] Financiamento FAPESP, chamada Auxílio à Pesquisa;
    \item[Processo:] \href{https://bv.fapesp.br/pt/auxilios/105047/um-sistema-universal-para-geracao-e-correcao-automatica-de-questoes-parametrizadas/}{2018/23561-1};
    \item[Coordenador:] Prof. Francisco de Assis Zampirolli.
    %\item[Membros da Equipe::] 
\end{description}

O Auxílio financeiro disponibilizado pela FAPESP ajudou em todas as publicações realizadas durante a vigência do projeto (ver lista de publicações disponibilizadas em \url{http://vision.ufabc.edu.br}). Além disso, o serviço de geração e correção automática de exames está disponível gratuitamente no \href{https://github.com/fZampirolli/mctest}{GitHub} e também está implantado para os professores da UFABC em \url{http://mctest.ufabc.edu.br}. 

\subsection{Participação em Projetos de Pesquisa}

\subsubsection{Conjunto de aplicativos móveis acessíveis para apoio ao aluno deficiente visual e auditivo}

\noindent\textbf{Resumo:} Este projeto tem como objetivo desenvolver um conjunto de aplicativos móveis (para Android) que se adaptam às necessidades dos estudantes com deficiência visual e auditiva, servindo de tecnologia assistiva no âmbito acadêmico. Inclui nesse conjunto: configurador/adaptador de acessibilidade, registro de aulas, assistente para provas e uso opcional de teclado tátil.

\begin{description}
    \item[Vigência:] Jul/2010 - Jun/2012;
    \item[Auxílio:] CAPES - Edital nº 03/2018 - Fomento à inovação para o desenvolvimento e aplicação de tecnologias de informação e comunicação em educação na temática ferramentas de acessibilidade;
    \item[Processo:] \href{https://www.gov.br/capes/pt-br/centrais-de-conteudo/24052018Edital32018Resultadofinal.pdf}{88887.185266/2018-00};
    \item[Coordenador:] Profa. Juliana Braga
    \item[Membros da Equipe:] \hfill 
    \begin{itemize}
        \item Prof. Francisco de Assis Zampirolli
        \item Profa. Marina Sparvol de Medeiros
        \item Profa. Priscila Benitez Afonso
        \item Prof. André Luiz Brandão
        \item Profa. Luciana Pereira
        \item Profa. Rafaela Vilela da Rocha
        \item Profa. Denise Goya
        \item Profa. Carla Lopes Rodriguez
        \item Profa. Mirtha Lina Fernandéz Venero
        \item Profa. Kate Mamhy Oliveira Kumada
        \item Prof. Edson Pinheiro Pimentel
        % \item Profa. Silva Dotta
        % \item Profa. Karla Vittori
        % \item Prof. Aiton Paulo de Oliveira Junior
        % \item Profa. Lucia Regina Horta Rodrigues Franco
    \end{itemize}
\end{description}

\subsubsection{Desenvolvimento de sistema de segurança para veículo autônomo em aplicação agrícola}

\noindent\textbf{Resumo:} O objetivo geral deste projeto é o desenvolvimento da tecnologia de automação veicular que leva um caminhão comercial de uso agrícola, que se encontra no nível de automação SAE-2, para o nível de automação SAE-3. Elevar a automação do caminhão para o nível SAE-3 aumentará a segurança de operação do equipamento, tornará o produto mais competitivo, aumentará as competências técnicas e científicas do grupo de trabalho, aumentará a transferência de conhecimentos entre os participantes e permitirá até mesmo a criação de \textit{startups} que ofereçam produtos para automação de outros equipamentos \textit{off-road} (ver \href{https://sites.google.com/view/offroad-autonomous/home?authuser=0}{portal do projeto}).

\begin{description}
    \item[Vigência:] Dez/2021 - Nov/2024;
    \item[Auxílio:] Fundação para o Desenvolvimento Tecnológico da Engenharia (FDTE);
    \item[Processo:] \href{https://pesquisa.in.gov.br/imprensa/servlet/INPDFViewer?jornal=530&pagina=154&data=24/12/2021&captchafield=firstAccess}{Rota 2030 - Linha V - PARCERIA FUNDEP - 27192.02.01/2021.02.00};
    \item[Coordenador:] Prof. Luis Antonio Celiberto Junior 
    \item[Membros da Equipe:] \hfill 
    \begin{itemize}
        \item Prof. Francisco de Assis Zampirolli
        \item Prof. Ugo Ibuski 
        \item Prof. Edson Caoru Kitani 
        \item Prof. Leopoldo Rideki Yoshioka 
    \end{itemize}
\end{description}

O Governo Federal, através do Decreto nº 9.557/18, instituiu o Programa Rota 2030 – Mobilidade e Logística para apoiar o desenvolvimento tecnológico, inovação e segurança veicular. O projeto em que participo integra grupos de pesquisa da UFABC, FATEC e Poli USP, com envolvimento de alunos de iniciação científica (5 alunos), mestrado (3 alunos) e doutorado (2 alunos).

\section{Comitê da SBC para elaborar os Referencias de Formação da Computação}

Participei do comitê para redigir os \href{https://sol.sbc.org.br/livros/index.php/sbc/catalog/book/63}{``Referenciais de Formação para os Cursos de Graduação em Computação no Brasil''} para o curso de Ciência da Computação \cite{2017:Calsavara.Serra.ea}. Também publicamos o artigo relacionado \cite{2018:Calsavara.Serra.ea}, premiado no \textit{Workshop} sobre Educação em Computação (WEI):

\begin{minipage}{0.8\linewidth}
\textbf{Método baseado nos Referenciais de Formação da SBC para reestruturação de descritivos de disciplinas de Ciência da Computação em conformidade com as DCN de 2016} -- Os  Referenciais  de  Formação  para  o  Bacharelado  em  Ciência  da Computação  (RF-CC-17)  da  SBC  organizam  as  competências  e  habilidades descritas  nas  \href{http://portal.mec.gov.br/index.php?option=+com_docman&view=download&alias=52101-rces005-16-pdf&category_slug=+novembro-2016-pdf&Itemid=30192}{Diretrizes  Curriculares  Nacionais  para  os  Cursos  de  Graduação em Computação (DCN16)} em eixos de formação e também indicam um conjunto de  conteúdos  associados.  Este  ensaio  apresenta  um  método,  baseado  nos  RF-CC-17, para elaborar um Mapeamento de Conformidade e Mobilização (MCM), como parte do descritivo de uma disciplina. Como exemplo, o método é aplicado na  elaboração  de  dois  descritivos  distintos  de  introdução  à  programação,  um baseado no paradigma imperativo e outro, orientado a objetos. Por fim, discute as  vantagens  de  se  usar  o  método  para  auxiliar  na  revisão  de  projetos pedagógicos de cursos vigentes tal que fiquem em conformidade com as DCN16.
\end{minipage}

\section{Revisor de eventos científicos}

\begin{itemize}
    \item Congresso Brasileiro de Engenharia Biomédica - CBEB: 2020;
    \item Congresso Brasileiro de Automática - CBA: 2016;
    \item Simpósio de IC/UFABC: 2010, 2011, 2012, 2013, 2014 e 2015;
    \item Feira Brasileira de Ciências e Engenharia - FEBRACE: 2010;
    \item Simpósio de Engenharia de Produção - SIMPEP: 2008 e 2009.
\end{itemize}

\section{Revisor de artigos em periódicos}

\begin{itemize}
    \item \textit{Proceedings of the IEEE};
    \item \textit{ACM Computing Surveys};
    \item \textit{Journal of Selected Topics in Signal Procesing}.
\end{itemize}

\section{Pareceres ad hoc}

\begin{itemize}
    \item Fundação de Amparo à Pesquisa do Estado de São Paulo - FAPESP (2008 - atual);
    \item Conselho Nacional de Desenvolvimento Científico e Tecnológico - CNPq  (2014 - atual).
\end{itemize}

\section{Eventos científicos}

\subsection{Apresentações de Trabalhos e Palestras}

\input{texLattes/Apresentacoes.tex}

\subsection{Participante em Eventos}

\input{texLattes/EventosParticipante.tex}

\subsection{Ouvinte em Eventos}

\input{texLattes/EventosOuvinte.tex}

% \subsection{Palestras}

% \begin{enumerate}
%     \item Zampirolli, F. A. \textit{Distance Transform: Algorithms and Applications}, 2014.  Evento: \href{http://www.wvc2014.facom.ufu.br/sites/wvc2014.facom.ufu.br/files/Anexos/01.10.14.WVC_2014.pdf#overlay-context=program}{WVC};
%     \item Zampirolli, F. A. Modelagem de Células usando Morfologia Matemática e Grafos de Vizinhança, 2008.  Evento: I Simpósio Docente da UFABC;
%     \item Zampirolli, F. A. Linhas de Projeto para Extensão - Iniciação Científica e Trabalhos de Conclusão de Curso, 2005.  Evento: VII Workshop da Graduação de Ciências Exatas e Tecnologia - Senac;
%     \item Zampirolli, F. A. \textit{Morphological Operators Based on Neighborhood Graph. An Extension of the MMach Toolbox}, 1996.  Evento: \textit{Brazilian Workshop'96 on Mathematical Morphology} - INPE.
% \end{enumerate}

\section{Participação em bancas de defesa}

\subsection{Bancas de Defesa de Mestrado}

\begin{enumerate}
    \item Zampirolli, F. A.; Prati, R. C; Selmini, A. M. Participação em banca de Kleber da Silva Pires. Análise de Resposta ao Tratamento Neoadjuvante em Câncer de Mama Utilizando Redes Profundas. 2022. Dissertação (Mestrado em Engenharia da Informação) - Universidade Federal do ABC; 
    \item Yoshioka, L. R.; Thomaz, C. E.; Zampirolli, F. A. Participação em banca de Michel André Lima Vinagreiro. Classificação Baseada na Análise dos Componentes Principais de Mapas de Características Gerados por Redes Neurais Convolucionais Profundas. 2021. Dissertação (Mestrado em Engenharia Elétrica) - Universidade de São Paulo;
    \item Botelho, W. T.; Zampirolli, F. A.; Edson, E. P. Participação em banca de Fabiana Naomi Iegawa. Aprendizado profundo aplicado em SLAM visual para identificar fechamento de loop. 2022. Dissertação (Mestrado em Engenharia da Informação) - Universidade Federal do ABC; 
    \item Zampirolli, F. A.; Cravo, A. G. S.; Sims, J. A. Participação em banca de João Carlos Pandolfi Santana. Aprendizado com Transferência em Imagens Retinográficas. 2021. Dissertação (Mestrado em Ciência da Computação) - Universidade Federal do ABC;
    \item Simões, S. N.; Komati, K. S.; Zampirolli, F. A.; Tello, R. J. M. G. Participação em banca de Rosana Aurélio de Jesus. Uma Investigação sobre a Importância dos Canais de Cores das Imagens de Fundoscopia nas Abordagens de Aprendizado de Máquina para Auxiliar na Identificação do Glaucoma. 2021. Dissertação (Mestrando em Computação Aplicada) - Instituto Federal do Espírito Santo (Serra);
    \item Zampirolli, F. A.; Edson, E. P.; Santos, Carlos S. Participação em banca de Daniel Gonçalves da Silva. Verificação Facial em Avaliações utilizando Redes Neurais Convolucionais Profundas. 2021. Dissertação (Mestrado em Engenharia da Informação) - Universidade Federal do ABC;
    \item Bevilacqua, J. S.; Zampirolli, F. A.; Roma Neto, E. Participação em banca de Erick Pereira Santos. Mineração de Dados aplicada à Tuberculose nos municípios do estado de São Paulo. 2020. Dissertação (Mestrado em Matemática Aplicada) - Universidade de São Paulo;
    \item Zampirolli, F. A.; Martins Jr, D. C.; Quilici-Gonzalez, J. A. Participação em banca de Lincoln Lima Souza Canabrava Mota. Redes Neurais Convolucionais Para Classificação de Imagens e Detecção de Objetos com Casos de Uso. 2020. Dissertação (Mestrado em Ciência da Computação) - Universidade Federal do ABC;
    \item Zampirolli, F. A.; Martins Jr, D. C.; Quilici-Gonzalez, J. A.. Participação em banca de Fábio Rezende de Souza. Semântica Distribucional Aplicada à Avaliação Didática. 2019. Dissertação (Mestrado em Ciência da Computação) - Universidade Federal do ABC;
    \item Delgado, K.V.; Tuesta, E.F.; Zampirolli, F. A.; Silva, A. A. B. Participação em banca de Victor Miranda Gonçalves Jatobá. Uma abordagem personalizada no processo de seleção de itens em Testes Adaptativos Computadorizados. 2018. Dissertação (Mestrado em Sistemas de Informação) - Universidade de São Paulo;
    \item Nascimento, M. Z.; Scott, A. L.;  Zampirolli, F. A. Participação em banca de Wagner Lopes Moreira Junior. Uma Nova Abordagem de Descritor de Textura Baseada em Transformada Ripplet para Classificação de Lesões da Mama. 2018. Dissertação (Mestrado em Engenharia da Informação) - Universidade Federal do ABC;
    \item Rodrigues, P. S. S.; Tonidandel, F.; Zampirolli, F. A. Participação em banca de Victor Henrique Conforto. Segmentação de imagens coloridas utilizando algoritmos bioinspirados. 2017. Dissertação (Mestrado em Engenharia Elétrica) - Centro Universitário da Fei;
    \item Neves, L. A.; Cansian, A. M.; Zampirolli, F. A. Participação em banca de Guilherme Freire Roberto. Percolação Multimensional e Multiescala para Quantificação de Imagens Histológicas de Linfomas. 2017. Dissertação (Mestrado em Ciência da Computação) - Universidade Estadual Paulista;
    \item Tonidandel, F.; Rodrigues, P. S. S.; Zampirolli, F. A. Participação em banca de Sidney Gitcoff Telles. Análise do desempenho de algoritmos para o reconhecimento de objetos aplicados em ambientes residenciais. 2017. Dissertação (Mestrado em Engenharia Elétrica) - Centro Universitário da Fei;
    \item Eler, M. M.; Zampirolli, F. A.; Cordeiro, D. A.; Soares, M. S.. Participação em banca de Robinson Crusoé da Cruz. Análise empírica sobre a influência das métricas CK na testabilidade de software orientado a objetos. 2017. Dissertação (Mestrado em Escola de Artes, Ciências e Humanidades) - Universidade de São Paulo;
    \item Silva, A. G.; Rosso Jr., R. S. U.; Zampirolli, F. A. Participação em banca de Marina Siva Fouto. Segmentação Automática da Zona Avascular Foveal em Imagens de Fundo de Olho para Auxílio ao Diagnóstico de Retinopatia Diabética. 2016. Dissertação (Mestrado em Computação Aplicada) - Universidade do Estado de Santa Catarina;
    \item Zampirolli, F. A. Participação em banca de Vinicius Eduardo Ferreira dos Santos Silva. Protocolo de negociação e colaboração baseado em ebXML a fim de otimizar processos portuários. 2015. Dissertação (Mestrado em Ciência da Computação) - Universidade Federal do ABC;
    \item Rodrigues, P. S. S.; Donato, G. H. B.; Zampirolli, F. A. Participação em banca de Fernando Azevedo Fardo. Análise de textura para detecção de trincas em corpos de prova da mecânica da fratura. 2015. Dissertação (Mestrado em Engenharia Elétrica) - Centro Universitário a Fei;
    \item Carvalho, M. A. G.; Zampirolli, F. A.; Rittner, L.. Participação em banca de Tiago Willian Pinto. Segmentação de imagens digitais combinando watershed e corte normalizado em grafos. 2014. Dissertação (Mestrado em TECNOLOGIA) - Universidade Estadual de Campinas;
    \item Spina, E.; Zampirolli, F. A.; Barco, Luiz. Participação em banca de Lucar Segismundo Moreno Lago. Fatores humanos na dependabilidade de sistemas de software desenvolvidos com práticas ágeis. 2014. Dissertação (Mestrado em Engenharia Elétrica) - Universidade de São Paulo;
    \item Rodrigues, P. S. S.; Maia, R. F.; Zampirolli, F. A. Participação em banca de Werner Fukuma. O potencial de redes complexas para análise do mercado de ações. 2013. Dissertação (Mestrado em Engenharia Elétrica) - Centro Universitário da Fei;
    \item Thomaz, C. E.; Aquino, P. T. Jr; Zampirolli, F. A. Participação em banca de André Sobiecki. Segmentação e restauração digital de artefatos em imagens frontais de face. 2012. Dissertação (Mestrado em Engenharia Elétrica) - Centro Universitário da Fei,
\end{enumerate}


\subsection{Bancas de Defesa de Doutorado}

\begin{enumerate}
    \item Santos Jr, A. R.; Ana, P. A.; Zampirolli, F. A.; Simoes, R.; Macedo, M. M. G. Participação em banca de Horácio Emidio de Lucca Junior. Classificação de imagens mamográficas por \textit{open source}: uma proposta de uso para auferir maior eficiência no diagnóstico de câncer. 2022. Tese (Doutorado em Biotecnociência) - Universidade Federal do ABC;
    \item Aquino, P. T. Jr; Gerra, E. M.; Zampirolli, F. A.; Bianchi, R. A. C.; Lopes, G.A.W. Participação em banca de Fábio Villamarin Arrebola. Modelo de sugestão de código-fonte baseado em modelos de linguagem organizados em um contexto hierárquico. 2018. Tese (Doutorado em Engenharia Elétrica) - Centro Universitário da FEI;
    \item Fabris, A. E.; Zampirolli, F. A.; Batista, V. R.; Nascimento, M. Z.; Gomes, J. N. V. Participação em banca de Ivana Soares Bandeira. Interfaces humano-computador aplicadas ao desenho de poligonais tridimensionais e à solução numérica do problema de Plateau. 2016. Tese (Doutorado em Matemática Aplicada) - Universidade de São Paulo;
    \item Lotufo, R. A.; Zampirolli, F. A.; Hirata Jr., R.; Tozzi, C. L.; Martino, J. M. Participação em banca de Alexandre Gonçalves Silva. Uso de Árvore de Componentes para Filtragem, Segmentação. 2009. Tese (Doutorado em Engenharia Elétrica) - Universidade Estadual de Campinas;
    \item Lorena, L. A. N.; Banon, G. J. F.; Carvalho, S. V.; Oliveira, J. R. F.; Hirata, N. S. T.; Zampirolli, F. A. Participação em banca de Sérgio Donizete Faria. Projeto de Operadores Morfológicos Parametrizados por Tabelas de Transformação de Níveis de Cinza. 2004. Tese (Doutorado em Computação Aplicada) - Instituto Nacional de Pesquisas Espaciais;
    \item Zampirolli, F. A.; Monteiro, A. M. V.; Banon, G. J. F.; Oliveira, J. R. F.; Fonseca, L. M. G.; Ferreira, M. G. V.; Furuie, S. S. Participação em banca de Juliana Cristina Braga. Procedência de Dados: Teoria e Aplicação ao Processamento de Imagens. 2004. Tese (Doutorado em Computação Aplicada) - Instituto Nacional de Pesquisas Espaciais.
\end{enumerate}

\subsection{Bancas de Qualificação de Mestrado}

\begin{enumerate}
    \item Botelho, W. T.; Zampirolli, F. A.; Edson, E. P. Participação em banca de Fabiana Naomi Iegawa. Aprendizado profundo aplicado em SLAM visual para identificar fechamento de loop. 2022. Exame de qualificação (Mestrando em Ciência da Computação) - Universidade Federal do ABC;
    \item Zampirolli, F. A.; Edson, E. P.; Santos, Carlos S. Participação em banca de Daniel Gonçalves da Silva. Verificação Facial em Avaliações utilizando Redes Neurais Convolucionais Profundas. 2021. Exame de qualificação (Mestrando em Ciência da Computação) - Universidade Federal do ABC;
    \item Yoshioka, L. R.; Thomaz, C. E.; Zampirolli, F. A. Participação em banca de Michel André Lima Vinagreiro. Classificação Baseada na Análise dos Componentes Principais de Mapas de Características Gerados por Redes Neurais Convolucionais Profundas. 2021. Exame de qualificação (Mestrando em Engenharia Elétrica) - Universidade de São Paulo;
    \item Selmini, A. M.; Prati, R. C; Zampirolli, F. A. Participação em banca de Kleber da Silva Pires. Análise de Resposta ao Tratamento Neoadjuvante em Câncer de Mama Utilizando Redes Profundas. 2021. Exame de qualificação (Mestrando em Ciência da Computação) - Universidade Federal do ABC;
    \item Simões, S. N.; Komati, K. S.; Zampirolli, F. A.; Tello, R. J. M. G. Participação em banca de Rosana Aurélio de Jesus. Uma Investigação sobre a Importância dos Canais de Cores das Imagens de Fundoscopia nas Abordagens de Aprendizado de Máquina para Auxiliar na Identificação do Glaucoma. 2021. Exame de qualificação (Mestrando em Computação Aplicada) - Instituto Federal do Espírito Santo (Serra);
    \item Santos, C. S.; Zampirolli, F. A.; Fantinato, D. G.; Covoes, T. F. Participação em banca de João Carlos Pandolfi Santana. Aplicações de Aprendizado com Transferência em Classificação de Imagens Médicas. 2020. Exame de qualificação (Mestrando em Ciência da Computação) - Universidade Federal do ABC;
    \item Kobayashi, G.; Braga, Juliana; Edson, E. P.; Zampirolli, F. A. Participação em banca de Paulo Cesar Angelo. Padrão para orquestração de servias BPEL aplicado ao modelo TISS de troca de informações na saúde suplementar. 2014. Exame de qualificação (Mestrando em Ciência da Computação) - Universidade Federal do ABC;
    \item Santos, C. S.; Zampirolli, F. A.; Martins Jr, D. C. Participação em banca de Renato Stoffalette João. Projeto de Operadores Morfológicos de Imagens com Aprendizado de Máquina. 2013. Exame de qualificação (Mestrando em Ciência da Computação) - Universidade Federal do ABC;
    \item Kobayashi, G.; Steinberger M. B.; Zampirolli, F. A. Participação em banca de Daniel Chinen Domingues. Reconhecimento de expressões faciais expontâneas por dispositivos móveis. 2013. Exame de qualificação (Mestrando em Engenharia da Informação) - Universidade Federal do ABC;
    \item Santana, F. S.; Zampirolli, F. A. Participação em banca de Reinaldo de Souza Gonzaga. Uma Abordagem Arquitetural para a Orquestração Dinâmica de Serviços. 2012. Exame de qualificação (Mestrando em Ciência da Computação) - Universidade Federal do ABC.
    \item Nascimento, M. Z.; Zampirolli, F. A. Participação em banca de Domingos Lucas Latorre de Oliveira. Método Computacional para Segmentação de Componentes Histológicos da Próstada. 2012. Exame de qualificação (Mestrando em Engenharia da Informação) - Universidade Federal do ABC;
    \item Kurashima, C. S.; Zampirolli, F. A. Participação em banca de Josivan Pereira da Silva. TV Digital Tridimensional: Sistema de Renderização de Cenas Imersivas. 2011. Exame de qualificação (Mestrando em Engenharia da Informação) - Universidade Federal do ABC.
    \item Nascimento, M. Z.; Zampirolli, F. A. Participação em banca de Rogério Daniel Dantas. Extração e Seleção de Atributos para a Classificação de Nódulos Mamários com Classificadores Fuzzy. 2010. Exame de qualificação (Mestrando em Engenharia da Informação) - Universidade Federal do ABC;
    \item Nascimento, M. Z.; Zampirolli, F. A. Participação em banca de Gabriel Paniz Patzer. Método Automático para Criação de Mapas Polares Baseados em Alinhamento de Imangens. 2010. Exame de qualificação (Mestrando em Engenharia da Informação) - Universidade Federal do ABC.
\end{enumerate}

\subsection{Monografias de cursos de aperfeiçoamento/especialização}

\begin{enumerate}
    \item Goya, D.; Zampirolli, F. A. Participação em banca de Sandreia de Almeida Moura Aquino. Os recursos pedagógicos acessíveis como ferramenta potente na educação infantil visando uma aprendizagem inclusiva. 2022. Monografia (Aperfeiçoamento/Especialização em Especialização em Educação Especial e Inclusiva) - Universidade Federal do ABC;
    \item Zampirolli, F. A. Participação em banca de Eliana Domingues da Cruz Milev. Experimento em Manutenção de Software. 2005. Monografia (Aperfeiçoamento/Especialização em Especialização Em Tecnologia da Informação) - Centro Universitário Senac;
    \item Zampirolli, F. A. Participação em banca de Wilson José Gama Junior. Tecnologias em Reconhecimento de Voz. 2005. Monografia (Aperfeiçoamento/Especialização em Especialização Em Tecnologia da Informação) - Centro Universitário Senac;
    \item Zampirolli, F. A. Participação em banca de Fábio Segato Martins. Controle de projetos de uma empresa júnior de comunicação. 2005. Monografia (Aperfeiçoamento/Especialização em Especialização Em Tecnologia de Objetos) - Centro Universitário Senac;
    \item Zampirolli, F. A. Participação em banca de Renata Prates Markert. Sistema Aluno Online. 2005. Monografia (Aperfeiçoamento/Especialização em Especialização Em Tecnologia de Objetos) - Centro Universitário Senac;
    \item Zampirolli, F. A.; Ana, A. P. G.; Almeida, E. S. Participação em banca de César Augusto de Lima Rossi. Criação de um Editor de Interface Gráfica. 2004. Monografia (Aperfeiçoamento/Especialização em Especialização Em Tecnologia de Objetos) - Centro Universitário Senac;
    \item Zampirolli, F. A.; Ana, A. P. G.; Almeida, E. S. Participação em banca de Daniel Filgueiras. Sistema de mensagem usando XML: SPB - Sistema de Pagamento Brasileiro. 2004. Monografia (Aperfeiçoamento/Especialização em Especialização Em Tecnologia de Objetos) - Centro Universitário Senac;
    \item Zampirolli, F. A.; Ana, A. P. G.; Almeida, E. S. Participação em banca de Bruno Inajá. Ferramenta de suporte a administração de banco de dados. 2004. Monografia (Aperfeiçoamento/Especialização em Especialização Em Tecnologia de Objetos) - Centro Universitário Senac.
\end{enumerate}

\subsection{Trabalhos de conclusão de curso de graduação}

\begin{enumerate}
    \item Josko, J. M.; Leite, S. C.; Zampirolli, F. A. Participação em banca de Renato de Avila Lopes. Aprendizado de Máquinas para a Previsão de Tendências de Séries Temporais Financeiras. 2022. Trabalho de Conclusão de Curso (Graduação em Ciência da Computação) - Universidade Federal do ABC;
    \item Zampirolli, F. A.; Bordin Jr, C. J.; Nagamuta, V. Participação em banca de Ana Paula Magalhães Silva.Utilizando ferramentas gratuitas para automatizar testes de software em portais web. 2019. Trabalho de Conclusão de Curso (Graduação em Engenharia de Informação) - Universidade Federal do ABC;
    \item Quilici-Gonzalez, J. A.; Mena-Chalco, J. P.; Zampirolli, F. A. Participação em banca de Fábio Rezende de Souza. Detecção e classificação de tipos de pólipos em coloscopia utilizando processamento de imagens. 2016. Trabalho de Conclusão de Curso (Graduação em Ciência da Computação) - Universidade Federal do ABC;
    \item Nagamuta, V.; Nomura, L.; Zampirolli, F. A. Participação em banca de Cássia de Souza Carvalho. Computação sensível a contexto: um estudo de caso na plataforma Android. 2012. Trabalho de Conclusão de Curso (Graduação em Ciência da Computação) - Universidade Federal do ABC;
    \item Quilici-Gonzalez, J. A.; Rozante, L.C.S.; Zampirolli, F. A. Participação em banca de André Xavier Martinez. Espelho digital para imagem simétrica incompleta e reposicionamento de faces - Aplicação em Processamento Digital de Imagens Usando CUDA. 2012. Trabalho de Conclusão de Curso (Graduação em Ciência da Computação) - Universidade Federal do ABC;
    \item Fraga, F. J.; Nascimento, M. Z.; Zampirolli, F. A. Participação em banca de Felipe Breve Siola. Desenvolvimento de um Software para Reconhecimento de Sinais em Libras através de Vídeo. 2010. Trabalho de Conclusão de Curso (Graduação em Bacharelado em Ciência da Computação) - Universidade Federal do ABC;
    \item Zampirolli, F. A. Participação em banca de Rodrigo da Paixão Silva Rogério. Implementação do MDA para Geração de Código Automático Usando Modelos. 2006. Trabalho de Conclusão de Curso (Graduação em Bacharelado em Ciência da Computação) - Centro Universitário Senac;
    \item Zampirolli, F. A. Participação em banca de Eduardo Tolino. Análise e desenvolvimento de componentes para plataforma E-Business. 2006. Trabalho de Conclusão de Curso (Graduação em Bacharelado em Ciência da Computação) - Centro Universitário Senac;
    \item Zampirolli, F. A.; Silva, E. A.; Miranda, F. R. Participação em banca de Leandro Batista de Oliveira. Programação Orientada a Aspectos em C++. 2005. Trabalho de Conclusão de Curso (Graduação em Bacharelado em Sistema de Informação) - Centro Universitário Senac.
\end{enumerate}

% \section{Inserção nacional e internacional}

% \subsection{Colaboração com Grupo de Pesquisa Internacional}

% \subsection{Colaboração com Grupo de Pesquisa Nacional}

\section{Publicações Científicas}

As apresentações científicas apresentadas nesta seção foram extraídas do \href{http://www.google.com/url?q=http%3A%2F%2Flattes.cnpq.br%2F4127260763254001&sa=D&sntz=1&usg=AOvVaw1HddQQ8MGMUNyg9ZdkRNak}{Currículo Lattes}. As 12 publicações completas realizadas em eventos científicos antes do ingresso na UFABC em 2008 também foram citadas no Capítulo \ref{cap:introducao} e nas Referências, no final deste Memorial (as demais publicações desta seção não estarão nas Referências, por considerar redundante, porém possuem \textit{links} para DOI - no estilo do Lattes - ou PDF, quando possível). As informações de Qualis para os periódicos foram extraídas do \textit{plugin} \href{chrome-extension://cobekobjpobenpjdggbpkkklkcfoinen/html/options.html}{QLattes} {\color{red} no dia XXX de outubro de 2023}. As informações de Qualis para as publicações completas publicadas em eventos científicos foram extraídas do portal \url{https://ppgcc.github.io/discentesPPGCC/pt-BR/qualis}. Os 36 artigos resumos publicados em anais de congressos entre 1996 e 2014, contidos no Lattes, não serão apresentados nesta seção.



%\subsection{Artigos Aceitos para Publicação em Periódicos}

\subsection{Artigos Completos em Periódicos}

\sethlcolor{Blue}

\input{texLattes/RevistasCOMPLETO.tex}

\subsection{Artigos Completos em Eventos Científicos com Revisão}

\input{texLattes/EventosCOMPLETO.tex}

\subsection{Artigos Completos em Eventos Científicos (Resumo)}

\input{texLattes/EventosResumo.tex}

\subsection{Capítulos de Livro}

\begin{itemize}
    \item Teubl, F.; Batista, V. R.; Zampirolli, F. A. MakeTests: A Flexible Generator and Corrector for Hardcopy Exams. In: Springer International Publishing. (Org.). Computer Supported Education. 1ed.: 2022, p. 293-315.
    \item Calsavara, A.; Ana, A. P. G.; Zampirolli, F. A.; Carvalho, L. S. G.; Jonathan, M.; Correia, R. C. M. Referenciais de Formação: Bacharelado em Ciência da Computação. In: Zorzo, A. F.; Nunes, D.; Matos, E.; Steinmacher, I.; Leite, J.; Araujo, R. M.; Correia, R.; Martins, S. (Org.). Referenciais de Formação para os Cursos de Graduação em Computação. 1ed.: 2017, p. 9-39.
\end{itemize}

\subsection{Livros}

\begin{itemize}
    \item Batista, V. R.; Zampirolli, F. A.; Botelho, H. M. An Evolver Program for Weighted Steiner Trees. 1. ed. Moldova: LAP Lambert, 2022. v. 1. 56p.
    \item Neves, R.; Zampirolli, F. A. Processando a informação: um livro prático de programação independente de linguagem. 1. ed. São Bernardo do Campo: EdUFABC, 2017. 192p.
    \item Quilici-Gonzalez, J. A.; Zampirolli, F. A. Sistemas inteligentes e mineração de dados. 1. ed. Assis | SP: Triunfal Gráfica e Editora, 2014. 150p.
\end{itemize}

\section{Programa de Computador Registrado}

\begin{itemize}
    \item WFRCP - Optmal Walk Finder for Robotic Cleaning of Reflection Pools. 2018. Patente: Programa de Computador. N. do registro: BR512018001232-9, data de registro: 24/07/2018, título: "WFRCP - Optmal Walk Finder for Robotic Cleaning of Reflection Pools" , Instituição de registro: INPI - Instituto Nacional da Propriedade Industrial.
    \item MCTest 4.G. 2018. Patente: Programa de Computador. N. do reg.: BR512018001202-7, data de registro: 24/07/2018, título: "MCTest 4.G" , Instituição de registro: INPI - Instituto Nacional da Propriedade Industrial.
    \item MCTest 4.0. 2016. Patente: Programa de Computador. N. do registro: BR51201600134-3, data de registro: 14/10/2016, título: "MCTest 4.0" , Instituição de registro: INPI - Instituto Nacional da Propriedade Industrial.
    \item Zampirolli, F. A.; Kobayashi, G. MCTest 3.0. 2015. Patente: Programa de Computador. N. do registro: BR512015001445-5, data de registro: 27/11/2015, título: "MCTest 3.0" , Instituição de registro: INPI - Instituto Nacional da Propriedade Industrial.
    \item Zampirolli, F. A.; China, R. T.; Neves, R. MCTest 2.0. 2015. Patente: Programa de Computador. N. do registro: BR512015001444-7, data de registro: 27/11/2015, título: "MCTest 2.0" , Instituição de registro: INPI - Instituto Nacional da Propriedade Industrial.
    \item Zampirolli, F. A.; Quilici-Gonzalez, J. A.; Neves, R. MCTest. 2013. Patente: Programa de Computador. N. do registro: BR51201300123-7, data de registro: 08/11/2013, título: "MCTest" , Instituição de registro: INPI - Instituto Nacional da Propriedade Industrial.
    \item Gomes, M. N.; Graphvs, M. J.; Maculan, N.; Gomes, F. N.; Palhano, A. W. C.; Santiago, C. P.; Frota, Y. A. M.; Zampirolli, F. A. Sistema SisGRAFO. 1991. Patente: Programa de Computador. N. do registro: 94006923, título: "Sistema SisGRAFO"
\end{itemize}

\section{Distinções Acadêmicas e Prêmios}

\begin{itemize}
    \item 2020 - Menção Honrosa pelo protótipo "Reconhecimento facial para validação de usuário durante um questionário no Moodle", Apps.edu (WCBIE 2020).
   \item 2019 - 2o lugar com o projeto: Conjunto de Aplicativos Móveis Acessíveis para Apoio ao Aluno Deficiente Visual e Auditivo, 2o Workshop de Inovação da Diretoria de Educação a Distância da CAPES.
   \item 2018 - Prêmio UFABC de Inovação - com o software MCTest 4.G - Geração Automática de Testes de Múltipla Escolha, com Questões e Alternativas Aleatórias, Útil para Correção de um Grande número de Testes, UFABC.
   \item 2018 - Menção Honrosa ao trabalho "Método baseado nos Referenciais de Formação da SBC para reestruturação de descritivos de disciplinas de Ciência da Computação em conformidade com as DCN de 2016", Workshop sobre Educação em Computação (WEI).
   \item 2018 - Prêmio UFABC de Inovação - com o software WFRCP - Optmal Walk Finder for Robotic Cleaning of Reflection Pools, UFABC.
   \item 2016 - Prêmio UFABC de Inovação - com o software MCTest 4.0 - Correção Automática de Testes de Múltipla Escolha, com Questões e Alternativas Aleatórias, Útil para Correção de um Grande número de Testes, UFABC.
   \item 2015 - Prêmio UFABC de Inovação - com o software MCTest 2.0 - Aplicativo Android para Correção Automática de Testes de Múltipla-Escolha, UFABC.
   \item 2015 - Prêmio UFABC de Inovação - com o software MCTest 3.0 em python - Correção Automática de Testes de Múltipla-Escolha, sem Interação com o Usuário, Útil para Correção de um Grande número de Testes, UFABC.
   \item 2014 - Segundo Lugar no Eixo de Comunicação e Informação: Correção Automática de Testes de Múltipla Escolha Utilizando Processamento de Imagem e Vídeo no Android, Simpósio de Iniciação Científica da UFABC.
   \item 2014 - Prêmio UFABC de Inovação - com o software MCTest em Matlab, UFABC.
   \item 2010 - Primeiro Lugar dos Temas Livres: Modelo Biomecânico Computadorizado Aplicado em Laringes Suínas, II Congresso Paulista de Otorrinolaringologia.
   \item 1996 - Primeiro lugar no Concurso de Software do IME, USP.
\end{itemize}
\section{Considerações Finais}






