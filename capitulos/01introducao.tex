% ----------------------------------------------------------
% Introdução 
% Capítulo sem numeração, mas presente no Sumário
% ----------------------------------------------------------

\chapter[Introdução]{Introdução}\label{cap:introducao}
%\addcontentsline{toc}{chapter}{Introdução}

Este capítulo apresenta a minha vida acadêmica no ensino superior, destacando formações na graduação, mestrado e doutorado, também a atuação profissional antes de ingressar em 2008 na Universidade Federal do ABC (UFABC).

\section{Apresentação da Carreira}

\subsection{Dados Pessoais}

\setlength{\fboxsep}{10pt}
\setlength{\fboxrule}{2pt}
\fbox{%
\begin{minipage}{0.8\linewidth}
\begin{description}
\item[Nome:] Francisco de Assis Zampirolli
\item[Dada de Nascimento:] 06 de Dezembro de 1968
\item[Ingresso no Serviço Público Federal:] 28 de Janeiro de 2008
\item[Endereço:] Centro de Matemática Computação e Cognição (CMCC) \\ Av. dos Estados, 5001, Sala 819-B -- Bloco B -- Bairro Bangu \\ CEP 09210-580 -- Santo André/SP 
\item[Telefone:] +55 11 4996-0078 
\item[Email:] fzampirolli@ufabc.edu.br
\end{description}
\end{minipage}}

\subsection{Formação Acadêmica}

Nesta seção apresento a minha formação acadêmica: graduação, mestrado e doutorado. Destaco que nesses três períodos sempre procurei participar de projetos de pesquisas em ciência da computação.

\subsubsection{Graduação}

Em 1987, resolvi sair de um curso técnico e me preparar para o vestibular. Em Cachoeiro do Itapemirim-ES cursei o primeiro e o segundo ano em contabilidade na Escola de 2º. Grau Aristeu Portugal Neves. O terceiro ano fiz integrado com um curso pré-vestibular no Colégio Jesus Cristo Rei. No ano seguinte estava cursando matemática na Universidade Federal do ES (UFES), em  Vitória-ES, único bacharelado no ensino público no estado que tinha ênfase em ciência da computação (o Bacharelado em Ciência da Computação na UFES foi criado em 1993).
\ 	
Em 1992, graduei-me em Bacharelado em Matemática, com ênfase em Ciência da Computação pela UFES. Durante a graduação, participei de duas Iniciações Científicas, com bolsa do CNPq, com orientação do prof. Dr. Francisco Negreiro Gomes:
``Seleção de Portfólios via Programação Estocástica Multiestágios com Heurísticas Especialistas'' e
``SisGRAFO, Um Sistema Gráfico de Otimização para Suporte à Decisão''.
\
Esta segunda Iniciação Científica resultou nas publicações apresentadas na Tabela \ref{tab:tabelaG}.
\
A minha participação no software SisGRAFO foi elaborar um ambiente gráfico na linguagem Pascal para visualização de rotas de veículos. No penúltimo ano da graduação participei também de um estágio na fundação da UFES envolvendo: 1. Desenvolvimento de programas utilizando a ``Unidade Gráfica do Turbo Pascal 6.0'', tendo como objetivo aprimorar interface com o usuário; 2. Digitação de textos no Editor \LaTeX; 3. Criação de Programas nas linguagens de programação Turbo C, Turbo Pascal 6.0 e Assembler.

\begin{table}[!ht]
   \centering
   \caption{Publicações referentes aos trabalhos realizados durante a graduação.}\label{tab:tabelaG}
\begin{tabular}{|c|c|c|}
\hline
\textbf{Tipo} & \textbf{Evento} & \textbf{Referência} \\ \hline
  Conferência & CLAIO &  \cite{1992:Gomes.Gomes.ea} \\ \hline
  Conferência & SBPO &  \cite{1992:Gomes.Gomes.ea*1} \\ \hline
\end{tabular}
\end{table}

As disciplinas de computação na graduação foram direcionadas para a área de otimização combinatória. Isto facilitou a construção do software para roteamento de veículos desenvolvido na iniciação científica. O prazer pelo desenvolvimento deste software científico e pelo ensino despertou em mim o desejo em continuar na vida acadêmica com um mestrado em ciência da computação. 

\subsubsection{Mestrado}

Em 1993, ingressei no mestrado no Instituto de Matemática e Estatística na Universidade de São Paulo (IME/USP), com bolsa de mestrado no CNPq. 
Em 1997 defendi o meu mestrado em Matemática Aplicada com ênfase em Ciência da Computação sob orientação do prof. Dr. Junior Barrera (título: ``Operadores Morfológicos Baseados em Grafos de Vizinhanças – Uma Extensão da \textit{MMach Toolbox}'') \cite{1997:Barrera.Zampirolli.ea}, com publicações relacionadas apresentadas na Tabela \ref{tab:tabelaM}.

\begin{table}[!ht]
   \centering
   \caption{Publicações referentes aos trabalhos realizados durante o mestrado.}\label{tab:tabelaM}
\begin{tabular}{|c|c|c|}
\hline
\textbf{Tipo} & \textbf{Evento} & \textbf{Referência} \\ \hline
  Conferência & BWMM &  \cite{1996:Zampirolli} \\ \hline
  Conferência & SIBGRAPI &  \cite{1997:Barrera.Zampirolli.ea} \\ \hline
  Revista & RIA &  \cite{2008:Zampirolli*1} \\ \hline
  Conferência & SPIE &  \cite{1998:Barrera.Terata.ea} \\ \hline
\end{tabular}
\end{table}

Durante o mestrado, em 2005 e 2006, também participei de um projeto de pesquisa com bolsa da Fundação USP, para desenvolver um OCR usando processamento de imagens. Essa pesquisa gerou a publicação do último artigo da Tabela \ref{tab:tabelaM}.

O software implementado durante o mestrado, que resultou na minha dissertação, obteve o primeiro lugar no Concurso de Software dos alunos do IME/USP. Este software fez uma extensão ao software \texttt{MMach}, iniciado pelo prof. Barrera. A \texttt{MMach} foi uma \textit{toolbox} que implementou os operadores de Morfologia Matemática no ambiente de processamento de imagens Khoros. Estendi esta \textit{toolbox} para manipular imagens formadas por grafos de vizinhança e apresentei como um curso no BWMM'97 \cite{zampirolli1997graph}. O software \texttt{MMach}, desenvolvido inicialmente para o Khoros, foi um dos primeiros softwares gratuitos em processamento de imagens distribuídos internacionalmente pela Web. A \texttt{MMach} foi desenvolvida  na linguagem de programação ANSI C e teve contribuições de vários pesquisadores após o início da década de 90, incluindo alunos de iniciação científica, mestrado, doutorado e pós-doutorado inicialmente orientados dos professores Barrera (IME/USP), Roberto de Alencar Lotufo (FEEC/UNICAMP) e Gerald Banon (INPE). 

\subsubsection{Doutorado}

O desenvolvimento da \texttt{MMach} foi iniciado em 1992 e foram geradas várias versões, principalmente quando mudava a versão do Khoros. Por volta de 1995 o prof. Lotufo ingressou na equipe para tornar a \texttt{MMach} independente de plataforma (e do Khoros) \cite{lotufo1997MMachLib}. 

A interação com o prof. Lotufo foi boa durante o mestrado e eu tinha o desejo de mudar de instituição para estimular a motivação, novos desafios, etc. Assim, foi um processo natural fazer o doutorado na Engenharia de Computação da UNICAMP. 

Em 1997, após o meu mestrado, fui trabalhar com o prof. Lotufo na UNICAMP em um projeto de pesquisa financiado pela \href{https://softexcps.org.br/}{SOFTEX} e foi fornecida uma bolsa do CNPq, modalidade DTI, onde trabalhei no ``Desenvolvimento de um ambiente para ensino a distância usando XML'' em parceria com o Centro de Pesquisas Renato Archer (CenPRA) de Campinas e a empresa \href{https://sdc.com.br/}{SDC Information System}. As pesquisas foram aplicadas na documentação do software \texttt{mmorph}, versão mais recente da \texttt{MMach}, estruturada em XML, com geração automática de código para as linguagens C, Matlab e Python, e também geração automática da documentação nos formatos TXT, HTML e \LaTeX.

Inúmeros usuários utilizam esses softwares, porém infelizmente foi descontinuado em sua última versão, incorporado ao AdessoWiki, coordenada pelo prof. Lotufo \cite{machado2011adessowiki,rittner2011adessowiki}.
\
Apesar de não ser mais possível utilizar a \texttt{MMach}, excelentes publicações estão disponíveis, como o livro de \citeonline{dougherty2003hands}, que utilizo em minhas aulas de Processamento de Imagens utilizando bibliotecas como OpenCV, adaptados no Google Colab ou Jupter Notebook.

Em 1998, ingressei como aluno regular do doutorado com bolsa da FAPESP sob a supervisão do prof. Lotufo. No final de 1998 terminei as disciplinas e intensifiquei as pesquisas do meu doutorado. Iniciei pesquisando os modelos (padrões) de programação de uma biblioteca de processamento morfológico de imagens e isto motivou a usar a Transformada de Distância (TD) como estudo de caso, pois possui um grande número de publicações. Com estes estudos foi possível reescrever a TD usando os modelos de programação paralelo, sequencial e por propagação usando a erosão morfológica, além de contribuir com novos algoritmos da TD por propagação, referenciados na Tabela \ref{tab:tabelaD}.
\
\begin{table}[!ht]
   \centering
   \caption{Publicações referentes aos trabalhos realizados durante o doutorado.}\label{tab:tabelaD}
\begin{tabular}{|c|c|c|}
\hline
\textbf{Tipo} & \textbf{Evento} & \textbf{Referência} \\ \hline
  Conferência & WAICV &  \cite{2000:Lotufo.Zampirolli} \\ \hline
  Conferência & SIBGRAPI &  \cite{2000:Zampirolli.Lotufo} \\ \hline
  Conferência & SIBGRAPI &  \cite{2000:Zampirolli.Lotufo.ea} \\ \hline
  Conferência & SIBGRAPI &  \cite{2001:Lotufo.Zampirolli} \\ \hline
  Conferência & SIBGRAPI &  \cite{2002:Lotufo.Falcao.ea} \\ \hline
  Conferência & SugarLoafPlop &  \cite{zampirolli2005algoritmos} \\ \hline
  Relatório & Senac &  \cite{zampirolli2003independent} \\ \hline
  Conferência & XATA &  \cite{2006:Zampirolli.Lotufo.ea} \\ \hline
\end{tabular}
\end{table}
\
A tese de doutorado possui um apêndice sobre uma proposta de linguagem independente, que gerou um relatório técnico e um artigo apresentados no final desta tabela.

Em 2002, fui convidado a trabalhar no Centro Universitário Senac ministrando aulas no Bacharelado em Ciência da Computação (BCC) recém criado, que me ofereceu a oportunidade de continuar fazendo pesquisas. Também iniciei ministrando algumas disciplinas no BCC do Centro Universitário da FEI, de São Bernardo do Campo.

Em 2003, defendi a tese em Engenharia Elétrica na FEEC/UNICAMP, sob orientação do prof. Lotufo, com o título: ``Transformada de Distância por Morfologia Matemática'' \cite{zampirolli2003transformada}.

\subsection{Atuação Profissional}\label{sec:profissionalAntes}

A minha experiência em docência iniciou com os cursos no Centro de Ensino de Computação do IME/USP em 1994 e 1995, onde ministrei o curso de extensão  ``Noções Básicas de Computação''.
\
Em 2004, participei de duas bancas de doutorado no INPE na área de morfologia matemática, orientadas pelo prof. Gerald Banon.
\
No Centro Universitário Senac e no Centro Universitário da FEI ministrei disciplinas de Algoritmos, Programação Orientada a Objetos, Engenharia de Software e Sistemas de Informação. 

Com dedicação de 40 horas semanais no Centro Universitário Senac foi possível realizar diversas atividades, como as apresentadas nas Tabelas 
\ref{tab:tabelaP} e \ref{tab:tabelaP-R}.
\
\begin{table}[!ht]
   \centering
   \caption{Atividades de orientações e participações em bancas realizadas no Centro Universitário Senac.}\label{tab:tabelaP}
\begin{tabular}{|c|c|c|c|}
\hline
\textbf{Tipo} & \textbf{Bolsa} & \textbf{Referência} & Ano\\ \hline
  BCC-IC & Institucional &  Lucas Padovani Trias & 2004 \\ \hline
  BCC-IC & Institucional &  Tiago Magalhães Vieira & 2005 \\ \hline
  BCC-IC & PIBIC/CNPq &  Lucas Padovani Trias & 2006 \\ \hline
  BCC-TCC            & - & 3 bancas & 2004-2006 \\ \hline
  Especialização & - & orientações de 3 TCC's & 2004 \\ \hline
  Especialização & - & orientações de 3 TCC's & 2005 \\ \hline
  Especialização & - & 7 bancas TCC's & 2004-2006 \\ \hline
\end{tabular}
\end{table}
\
Atuei nas orientações e nas participações de bancas de trabalhos de conclusão de curso (TCC) na graduação do BCC e em pós-graduações (lato sensu): ``Especialização em Tecnologia de Objetos'' e ``Especialização em Tecnologia da Informação'', ver Tabela \ref{tab:tabelaP}. As orientações na pós-graduação foram desenvolvidas usando o processo RUP (\textit{Rational Unified Process}) para o desenvolvimento de software.
\
Ministrei também um curso de verão de ``Introdução ao RUP'' no Centro Universitário Senac. O RUP é um processo em hipertexto para auxiliar o desenvolvimento de software. 
\
De 2004 a 2006, coordenei a linha da pesquisa ``Computação Científica'', contendo alguns projetos, dentre eles, coordenei o projeto ``Sistema de Documentação de Software Web''. Este projeto foi inspirado na linguagem intermediária definida durante o meu doutorado (ver Apêndice de \citeonline{zampirolli2003transformada}). Essa linguagem tem como objetivo gerenciar os artefatos produzidos durante o processo de desenvolvimento de software seguindo a teoria de engenharia de software. 
\
Participei do Comitê de Ética, do Conselho de Curso e do Comitê de Iniciação Científica. Também assumi a coordenação do Bacharelado em Ciência da Comutação entre 2006 e 2007.
\
Após o doutorado, participei das seguintes publicações (resumos) apresentadas na Tabela \ref{tab:tabelaP-R}. 
\
Os artigos completos estão apresentados na Tabela \ref{tab:tabelaP-C}, observando que os três últimos artigos também estão referenciados na Tabela \ref{tab:tabelaM}, pois são referentes também ao meu mestrado, mas foram produzidos com suporte do Centro Universitário Senac.

\begin{table}[!ht]
   \centering
   \caption{Publicações (resumos) referentes aos trabalhos realizados no Centro Universitário Senac.}\label{tab:tabelaP-R}
\begin{tabular}{|c|c|c|}
\hline
\textbf{Tipo} & \textbf{Evento} & \textbf{Referência} \\ \hline
  Encontro & Senac-EPE &  \cite{trias2005toolbox} \\ \hline
  Simpósio & SIICUSP &  \cite{2005:Trias.Zampirolli} \\ \hline
  Congresso & CNMAC &  \cite{trias2005toolbox3} \\ \hline
  Encontro & Senac-EPE &  \cite{zampirolli2005ads} \\ \hline
  Encontro & Senac-EIC &  \cite{trias2006scripts} \\ \hline
  Conferência & X-Meeting &  \cite{zampirolli2007modeling} \\ \hline
  Conferência & X-Meeting &  \cite{rizzio2007multiscale} \\ \hline
\end{tabular}
\end{table}

\begin{table}[!ht]
   \centering
   \caption{Publicações (artigos completos) referentes aos trabalhos realizados no Centro Universitário Senac.}\label{tab:tabelaP-C}
\begin{tabular}{|c|c|c|}
\hline
\textbf{Tipo} & \textbf{Evento} & \textbf{Referência} \\ \hline
  Workshop & WEI &  \cite{yamamoto2005interdisciplinaridade} \\ \hline
Conferência & SugarLoafPlop &  \cite{zampirolli2005algoritmos} \\ \hline
  Relatório & Senac &  \cite{zampirolli2003independent} \\ \hline
  Conferência & XATA &  \cite{zampirolli2006independent} \\ \hline 
\end{tabular}
\end{table}


Participei também em 2007 de uma colaboração em pesquisa no IME/USP, sob a coordenação do prof. Barrara, retomando as pesquisas de processamento de imagens usando grafos, do meu mestrado, aplicadas agora na área de bioinformática. Conseguimos com essas pesquisas as duas últimas publicações da Tabela \ref{tab:tabelaP-R} e também o artigo \citeonline{zampirolli2010segmentation}, já como professor da UFABC.

\subsection{Eventos e Cursos}

Durante a minha formação e atuação profissional participei de vários eventos apresentados na Tabela \ref{tab:tabelaEventos}. Durante a minha formação acadêmica foram 15 eventos; Durante a minha atuação profissional no Centro Universitário Senac e no Centro Universitário da FEI foram 14 eventos; e após o ingresso na UFABC foram 17 eventos.

\begin{table}[!ht]
   \centering
   \caption{Participações em eventos.}\label{tab:tabelaEventos}
\begin{tabular}{|c|c|c|c|}
\hline
\textbf{Período} & \textbf{Ouvinte} & \textbf{Palestrante} & \textbf{Descrição} \\ \hline
  1989-2003 & 10 &  5 & Formação Acadêmica \\ \hline
  2004-2007 & 10 &  4 & Senac e FEI \\ \hline
  2008-2023 & 1 &  16 & Após ingresso na UFABC\\ \hline
\end{tabular}
\end{table}

Antes de ingressar na graduação, fiz um curso de programação Basic de 48 horas no ``Info-Center do Brasil (ICB)'', em Cachoeiro do Itapemirim-ES.
\
Durante a minha graduação fiz dois cursos de extensão em programação.
\
Com o incentivo do Centro Universitário Senac fiz quatro cursos. Todos esses cursos estão referenciados na Tabela \ref{tab:tabelaCursos}.

\begin{table}[!ht]
   \centering
   \caption{Participações em cursos extracurriculares.}\label{tab:tabelaCursos}
\begin{tabular}{|c|c|c|c|}
\hline
\textbf{Período} & \textbf{Duração (horas)} & \textbf{Instituição} & \textbf{Descrição} \\ \hline
  1986 & 48 &  ICB & Programação Basic \\ \hline
  1991 & 15 &  UFES & Programação Orientada a Objetos \\ \hline
  1992 & 42 &  UFES & Programação C \\ \hline
  2004 & 4 &  SOFTEX & mpsBR \\ \hline
  2005 & 32 &  SulSoft & ENVI \\ \hline
  2006 & 40 &  INPI & Propriedade Intelectual \\ \hline
  2007 & 24 &  INEP & Capacitação BASis \\ \hline
  2012 & 8  & Thomson Reuters & \textit{Web of Knowledge} \\ \hline
\end{tabular}
\end{table}

\subsection{Dados na Web}

\begin{itemize}
    \item \href{https://sites.google.com/site/fzampirolli}{Página Pessoal}
    \item \href{https://sig.ufabc.edu.br/sigaa/public/docente/portal.jsf?siape=1600876}{Página Pessoal no SIGAA}
    \item \href{http://lattes.cnpq.br/4127260763254001}{Currículo Lattes}
    \item \href{https://scholar.google.com.br/citations?hl=pt-BR&user=9jboxEoAAAAJ}{Google Acadêmico}
    \item \href{https://www.scopus.com/authid/detail.uri?authorId=56018682900}{Scopus}
    \item \href{https://www.webofscience.com/wos/author/record/D-2301-2012}{MyResearcherID}
    \item \href{https://orcid.org/0000-0002-7707-1793}{ORCID}
    \item \href{https://vcad-vision.ufabc.edu.br}{Laboratório VCAD}
\end{itemize}

%\newpage

\section{Contexto Legal da Promoção à Classe E}

A \href{https://cmcc.ufabc.edu.br/images/resolucao_027.pdf}{RESOLUÇÃO Nº 27 / 2022 - ConCMCC}  traz em sua ementa: ``Regulamenta os procedimentos no âmbito do CMCC para a promoção de docentes à Classe E, com denominação de Professor Titular, da Carreira do Magistério Superior da UFABC''. Por essa resolução, tenho que formalizar a solicitação de promoção encaminhando toda documentação definida na Resolução ConsUni Nº 161 para a direção do CMCC no prazo máximo de quatro meses antes do meu interstício, ou seja, após 28 de Setembro de 2023 (minha última progressão funcional para Associado 4 ocorreu em 28 de Janeiro de 2022). Após isso, a direção encaminhará o pedido para a Comissão de Promoção à Classe E do CMCC (CPCE-CMCC) em até sete dias. A CPCE-CMCC deverá encaminhar em até 30 dias uma sugestão de oito membros para compor a Comissão Especial de Avaliação, sendo três docentes titulares da UFABC como membros internos e cinco docentes titulares externos oriundos da minha área de conhecimento requerida nessa solicitação, que é 
\
\textbf{Processamento de Imagens e Visão Computacional}. 
\
Dessa sugestão de nomes o ConCMCC encaminhará para a Comissão de Vagas os nomes dos membros titulares (um interno e três externos) e os suplentes.

Contextualizando o Plano de Carreira do Magistério Superior vigente, a \href{https://www2.camara.leg.br/legin/fed/lei/2013/lei-12863-24-setembro-2013-777081-publicacaooriginal-141211-pl.html}{LEI Nº 12.863, 24.09.2013} altera a \href{https://www.planalto.gov.br/ccivil_03/_ato2011-2014/2012/lei/l12772.htm}{LEI Nº 12.772, 28.12.2012} e estrutura em classes A, B, C, D e E. A classe D (antigo Associado) possui três \textit{progressões} em quatro níveis. Chamamos de \textit{promoção} quando mudamos de classe e a classe E é a última a ser solicitada, após o professor estar por dois anos no Associado IV.

A \href{https://apur.org.br/wp-content/uploads/2013/10/PORTARIA_982.pdf}{PORTARIA MEC Nº 982, 03.10.2013},  traz em sua ementa: ``Estabelece as diretrizes gerais para fins de promoção à Classe E, com denominação de Professor Titular da Carreira do Magistério Superior e classe de Titular da Carreira de Magistério do Ensino Básico, Técnico e Tecnológico das Instituições Federais de Ensino vinculadas ao Ministério da Educação''. Esta portaria traz em seu segundo artigo:
\
\begin{quoting}[rightmargin=0cm,leftmargin=2cm]
{\footnotesize 
Art. 2º A promoção para a classe E, com denominação de Professor Titular da Carreira do Magistério Superior, dar-se-á observando os critérios e requisitos instituídos conforme inciso IV do §3º do artigo 12º da \href{https://www.planalto.gov.br/ccivil_03/_ato2011-2014/2012/lei/l12772.htm}{LEI Nº 12.772, 28.12.2012}:\\
I - possuir o título de doutor;\\
II - ser aprovado em processo de avaliação de desempenho; e\\
III - lograr aprovação de memorial que deverá considerar as atividades de ensino, pesquisa, extensão, gestão acadêmica e produção profissional relevante, ou defesa de tese acadêmica inédita.\\
Parágrafo único. A promoção ocorrerá observado o interstício mínimo de 24 (vinte e quatro) meses no último nível da classe D, com denominação de professor Associado.  
}
\end{quoting}

Considerando a \href{https://www.ufabc.edu.br/images/consuni/resolucoes/resolucao_consuni_161_-_dispoe_sobre_o_estabelecimento_aplicacao_de_criterios_para_avaliacao_de_docentes_para_professor_titular_de_carreira.pdf}{RESOLUÇÃO CONSUNI Nº 161, 07.01.2016}, que estabelece os critérios para avaliação de docentes com vistas ao acesso à Classe E, estabelece:
\
\begin{quoting}[rightmargin=0cm,leftmargin=2cm]
{\footnotesize 
Art. 1º A promoção funcional para a Classe E, com denominação de Professor Titular de Carreira do Magistério Superior da UFABC, na forma estabelecida pela Lei vigente, dar-se-á, desde que o requerente preencha cumulativamente os seguintes requisitos:\\
I - possuir o título de doutor;\\
II - ter cumprido o interstício mínimo de 24 (vinte e quatro) meses no último nível da Classe D, com denominação de Professor Associado IV;\\
III - ser aprovado em processo de Avaliação de Desempenho composto por: análise de \textbf{Mapa de Pontuação} e \textbf{Prova de Erudição}; e\\
IV - lograr aprovação de \textbf{Memorial} ou defesa de Tese Acadêmica Inédita.
}
\end{quoting}
\
Além disso, em seus artigos 4º (§2º e §3º) e 5º:
\
\begin{quoting}[rightmargin=0cm,leftmargin=2cm]
{\footnotesize 
§2º A \textbf{Prova de Erudição} será realizada na forma de uma conferência que visa demonstrar a excelência, competência e qualificação do requerente na área pleiteada.\\
§3º A \textbf{Prova de Erudição} deverá versar sobre tema proposto pelo requerente, relativo a sua área de atuação, tratando de suas contribuições do Estado da Arte e da Produção Bibliográfica Contemporânea, que seja relevante e que inclua perspectivas futuras.\\
Art. 5º O \textbf{Memorial} será baseado em exposição escrita das atividades do requerente relacionadas a ensino, pesquisa, extensão e gestão acadêmica, além de plano de ações que inclua perspectivas futuras e sua defesa deverá ser apresentada oralmente pelo requerente.
}
\end{quoting}




\section{Organização do Memorial}

Iniciei esse Memorial no Capítulo \ref{cap:introducao}, com a apresentação da minha formação acadêmica na graduação, mestrado e doutorado. Apresentei de forma resumida os projetos de pesquisa e as publicações desenvolvidas em cada uma dessas três fases. Também apresentei a minha atuação profissional nos Centros Universitários Senac e FEI, antes de ingressar na UFABC. Finalmente apresentei os aspectos legais vigentes para essa minha solicitação de promoção à Classe E.

No Capítulo \ref{cap:ensino} apresentei as minhas atividades de ensino na UFABC na graduação e na pós-graduação. Incluí também as orientações realizada na UFABC (no Centro Universitário Senac orientei apenas três iniciações científicas e também orientações em especializações).

O Capítulo \ref{cap:pesquisa}

O Capítulo \ref{cap:extensao}

O Capítulo \ref{cap:admin}

O Capítulo \ref{cap:conclusao}


Esse memorial não apresenta os comprovantes, apesar de muitos estarem disponíveis e referenciados em forma de \textit{links} (cor azul no texto). Muitos destes comprovantes foram enviados à CPPD em progressões/promoções anteriores. Os comprovantes referentes ao Mapa de Pontuação do interstício no último nível da Classe D estão disponíveis nesse processo de promoção para a Classe E.



