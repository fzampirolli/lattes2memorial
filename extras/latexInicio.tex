% ---------------------------------------------------------------
% Adaptado de https://pt.overleaf.com/latex/templates/template-ufabc-dissertacao/zmwgdkcsrxjb
% por Francisco de Assis Zampirolli, 2023
% ---------------------------------------------------------------

% ---------------------------------------------------------------
% Modelo LaTex para dissertação e tese do programa de Pós Graduação em Ciência da Computação da UFABC
% ---------------------------------------------------------------

\documentclass[
	% -- opções da classe memoir --
	12pt,					% tamanho da fonte
	openright,				% capítulos começam em pág ímpar (insere página vazia caso preciso)
	oneside,					% para impressão em verso e anverso. Oposto a oneside
	a4paper,					% tamanho do papel. 
	% -- opções da classe abntex2 --
	%chapter=TITLE,			% títulos de capítulos convertidos em letras maiúsculas
	%section=TITLE,			% títulos de seções convertidos em letras maiúsculas
	%subsection=TITLE,		% títulos de subseções convertidos em letras maiúsculas
	%subsubsection=TITLE,	% títulos de subsubseções convertidos em letras maiúsculas
	% -- opções do pacote babel --
	english,					% idioma adicional para hifenização
	%french,					% idioma adicional para hifenização
	%spanish,				% idioma adicional para hifenização
	brazil					% o último idioma é o principal do documento
	]{abntex2}

% ---------------------
% Pacotes OBRIGATÓRIOS
% ---------------------
\usepackage{lmodern}				% Usa a fonte Latin Modern			
\usepackage[T1]{fontenc}			% Selecao de codigos de fonte.
\usepackage[utf8]{inputenc}		% Codificacao do documento (conversão automática dos acentos)
\usepackage{lastpage}			% Usado pela Ficha catalográfica
\usepackage{indentfirst}			% Indenta o primeiro parágrafo de cada seção.
\usepackage{color}				% Controle das cores

\usepackage{graphicx,graphicx}	% Inclusão de gráficos
\usepackage{epsfig,subfig}		% Inclusão de figuras
\usepackage{microtype} 			% Melhorias de justificação
% ---------------------

% Colors package
\usepackage[dvipsnames]{xcolor}
\usepackage {soul}
\setul{0.3ex}{0.2ex}
 
% ---------------------
% Pacotes ADICIONAIS
% ---------------------
\usepackage{lipsum}						% Geração de dummy text
\usepackage{amsmath,amssymb,mathrsfs}	% Comandos matemáticos avançados 
\usepackage{setspace}  					% Para permitir espaçamento simples, 1 1/2 e duplo
\usepackage{verbatim}					% Para poder usar o ambiente "comment"
\usepackage{tabularx} 					% Para poder ter tabelas com colunas de largura auto-ajustável
\usepackage{afterpage} 					% Para executar um comando depois do fim da página corrente
\usepackage{url} 						% Para formatar URLs (endereços da Web)
% ---------------------

\usepackage{quoting}

\usepackage{hyperref}

% ---------------------
% Pacotes de CITAÇÕES
% ---------------------
\usepackage[brazilian,hyperpageref]{backref}	% Paginas com as citações na bibl
%\usepackage[alf,abnt-etal-cite=4]{abntex2cite}				% Citações padrão ABNT (alfa)
\usepackage[alf,abnt-etal-cite=4,abnt-etal-list=0,abnt-etal-text=emph]{abntex2cite}
%\usepackage[num]{abntex2cite}				% Citações padrão ABNT (numericas)
%\usepackage[style=alphabetic,sorting=nyt,sortcites=true,maxbibnames=9,autopunct=true,babel=hyphen,hyperref=true,abbreviate=false]{biblatex}
% ---------------------


% Configurações de CITAÇÕES para abntex2
\input{extras/conf_citacoes}

% Inclusão de dados para CAPA e FOLHA DE ROSTO (título, autor, orientador, etc.)
\input{extras/dados}

% Inclui Configurações de aparência do PDF Final
\input{extras/conf_pdf}

% O tamanho da identação do parágrafo é dado por:
\setlength{\parindent}{1.3cm}

% Controle do espaçamento entre um parágrafo e outro:
\setlength{\parskip}{0.2cm}  % tente também \onelineskip

% ---------------------
% Compila o índice
% ---------------------
\makeindex
% ---------------------

%%%%%%%%%%%%%%%%%%%%%%%%%%%
%%  INICIO DO DOCUMENTO  %%
%%%%%%%%%%%%%%%%%%%%%%%%%%%
\begin{document}

% Retira espaço extra obsoleto entre as frases.
\frenchspacing

% ----------------------------------------------------------
% ELEMENTOS PRÉ-TEXTUAIS (Capa, Resumo, Abstract, etc.)
% ----------------------------------------------------------
\pretextual

% Capa
\include{pretextual/capa}

% % Folha de rosto (o * indica que haverá a ficha bibliográfica)
% \imprimirfolhaderosto*

% % Imprimir Ficha Catalografica
% \include{pretextual/catalografica}

% % Inserir Folha de Aprovação
% \include{pretextual/assinaturas}

% % Dedicatória
% % ---
% Dedicatória
% ---
\begin{dedicatoria}
   \vspace*{\fill}
   \centering
   \noindent
   \textit{Dedico este ...
} \vspace*{\fill}
\end{dedicatoria}
% ---

% % Agradecimentos
% % ---
% Agradecimentos
% ---
\begin{agradecimentos}

Agradeço ...

\end{agradecimentos}
%% ---

% % Epígrafe
% % ---
% Epígrafe
% ---
\begin{epigrafe}
    \vspace*{\fill}
	\begin{flushright}
		\textit{``....''\\
		          (Autor)}
	\end{flushright}
\end{epigrafe}
% ---


% Resumo e Abstract
% ---
% RESUMOS
% ---

% RESUMO em português
\setlength{\absparsep}{18pt} % ajusta o espaçamento dos parágrafos do resumo
\begin{resumo}
\input{texLattes/Resumo.tex} \\
 \textbf{Palavras-chaves}: latex. abntex. editoração de texto.
\end{resumo}

% % ABSTRACT in english
% \begin{resumo}[Abstract]
%  \begin{otherlanguage*}{english}
%    This is the english abstract.

%    \vspace{\onelineskip}
 
%    \noindent 
%    \textbf{Keywords}: latex. abntex. text editoration.
%  \end{otherlanguage*}
% \end{resumo}

% Lista de ilustrações
\pdfbookmark[0]{\listfigurename}{lof}
\listoffigures*
\cleardoublepage

% Lista de tabelas
\pdfbookmark[0]{\listtablename}{lot}
\listoftables*
\cleardoublepage

% Lista de abreviaturas e siglas
% \begin{siglas}
%   \item[FAPESP] Fundação de Amparo à Pesquisa do Estado de São Paulo
%   \item[FEEC] Faculdade de Engenharia Elétrica e de Computação
%   \item[CMCC] Centro de Matemática, Computação e Cognição
%   \item[CNPq] Conselho Nacional de Desenvolvimento Científico e Tecnológico
%   \item[CPCE] Comissão de Promoção à Classe E
%   \item[CPPD] Comissão Permanente de Pessoal Docente
%   \item[IME] Instituto de Matemática e Estatística
%   \item[INEP] Instituto Nacional de Estudos e Pesquisas Educacionais Anísio Teixeira
%   \item[MEC] Ministério da Educação
%   %\item[MMach] Mathematical Morphology toolbox for the KHOROS
%   %\item[SisGRAFO] Sistema Gráfico de Otimização para Suporte
% à Decisão
%   \item[UFABC] Universidade Federal do ABC
%   \item[UFES] Universidade Federal do Espírito Santo
%   \item[UNICAMP] Universidade Estadual de Campinas
%   \item[USP] Universidade de São Paulo
% \end{siglas}

% % Lista de símbolos
% \begin{simbolos}
%   \item[$ \Gamma $] Letra grega Gama
%   \item[$ \Lambda $] Lambda
%   \item[$ \zeta $] Letra grega minúscula zeta
%   \item[$ \in $] Pertence
% \end{simbolos}

% Inserir o SUMÁRIO
\pdfbookmark[0]{\contentsname}{toc}
\tableofcontents*
\cleardoublepage

% ----------------------------------------------------------
% ELEMENTOS TEXTUAIS (Capítulos)
% ----------------------------------------------------------
\textual
% Elementos textuais com numeração arábica
\pagenumbering{arabic}
% Reinicia a contagem do número de páginas
\setcounter{page}{1}

\sethlcolor{Blue}

% Inclui cada capitulo 
